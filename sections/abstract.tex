\begin{abstract}
As an abstraction of human body estimation, human pose estimation is a demanded but challenging research topic in computer vision. The applications range from simple use in mobile devices or entertainment systems up to medical purposes. Therefore, an estimation approach must provide accurate results to be considered valuable for its area of application. For this reason, this overview of human pose estimation first explains datasets relevant to both benchmarking and supervision approaches, followed by computational rules for measuring the quality of an estimate. This field of research is not only very broad, but can also be viewed from many different perspectives, which is why a general distinction is made between the dimensions of the estimates. To provide essential knowledge for the approaches of later sections, the parametrized human body model \emph{smpl} is introduced and its principal concepts are subsequently documented. In 2d pose estimation, top-down and bottom-up frameworks were explained where deep neural networks aid in determining the pose. Important papers in the topic of 3d human pose estimation are presented and their architectures and results explained. Finally the current state of the broad topic human pose estimation is summarized with an outlook into possible future research.
\end{abstract}
