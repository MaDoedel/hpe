\section{models in human estimation}
This section is dedicated to human models, which are besides being a representative concept, also central aspects on which iterative pose estimation approaches are based. While simple stick models, where the human structure is modeled by a few vertices, show the joints and limbs in 2D estimation, 3D models, originally built from simple primitives such as cylinders, have evolved into a vertex structure showing the shape, muscles and body structure of a human. To conclude, this section explained in detail the human body models commonly used in 3D estimation methods.

\subsection{Skinned Multi-Person Linear model SMPL}
\label{sec:SMPL}
In \emph{SMPL} \cite{smpl} a model $M(\vec{\beta},\vec{\theta},\phi)$ is learned from the \emph{Caesar} 3d scans explained in \autoref{sec:benchmarks}, that returns a mesh from the input. This formulation is also included in \autoref{eq:model}, where $\mathbb{R}^{3N}$ is a vector of \emph{N = 6890} vertices sculpturing the mesh. In this formula, $\vec{\beta}$ is a vector that contains the coefficients for the shape blend shapes, while $\vec{\theta}$ is a vector of poses, e.g. the concatenated joint rotation matrix elements of all joints, which are also used as coefficients for pose blend shapes, and $\phi$ describes other learned parameters.

\begin{equation}
\label{eq:model}
M(\vec{\beta},\vec{\theta},\phi) : \mathbb{R}^{\vert \vec{\theta} \vert \times \vert \vec{\beta} \vert} \mapsto \mathbb{R}^{3N}
\end{equation}

\emph{SMPL} is based on vertex skinning and blend shapes. A vertex changes its position depending on the motion of the associated joint. This displacement is controlled by assigned blend weights. A vector $T \in \mathbb{R}^{3N}$ of vertex positions describes a gender neutral initial human model, while a matrix $W \in \mathbb{R}^{N \times K}$ represents the blend weights per vertices and \emph{K = 23} joints. The joints that describe the human structure and form the skeleton are represented by 3d psitions. Moreover, T can be rearranged by the pose-blending function $B_{P}(\vec{\theta})$ according to the given poses, leaving its initial shape unaffected, while $B_{S}(\vec{\beta})$ reshapes the identity model by its given shape blends. A skinning function $W(T,J,\vec{\theta},\mathcal{W})$ transforms the initial model $T$ by the joint positions $J \in \mathbb{R}^{3K}$ and rotations $\vec{\theta}$ of the desired model. The simplest skinning approach is called \emph{linear blend skinning}, where each resulting vertex $\bar{t}_{i}'$ depends on the sum of its weights $w_{k,i}$ per joint multiplied by its previous position $\bar{t}_{i}$ and a target destination, as documented in \autoref{eq:lbs}. 

\begin{equation}
\label{eq:lbs}
\bar{t}_{i}' = \sum_{k=1}^{K} w_{k,i} \cdot G_{k}'(\vec{\theta},J) \cdot \bar{t}_{i}
\end{equation}

Unfortunately, this approach is error-prone and requires additional refinement by the artists to reduce artifacts. To overcome these problems, the skinning method in \emph{SMPL} introduces new functions $T_{F}(\vec{\beta},\vec{\theta})$ and $J(\vec{\beta})$ so that \autoref{eq:model} and $W(T,J,\vec{\theta},\mathcal{W})$ can be reformulated by \autoref{eq:smpl}.

\begin{equation}
\label{eq:smpl}
M(\vec{\beta},\vec{\theta}) = W(T_{F}(\vec{\beta},\vec{\theta}),J(\vec{\beta}),\vec{\theta},\mathcal{W}) \mapsto \mathbb{R}^{3N}
\end{equation}

In $T_{F}(\vec{\beta},\vec{\theta})$, the initial body $T$ is considered as a known parameter, while the pose correctives $B_{P}(\vec{\theta})$ and the reshapes $B_{S}(\vec{\beta})$ are added. Since the focus is on pose estimation, this section explains $B_{P}(\vec{\theta})$ in more detail. With respect to \autoref{eq:Tf}, $B_{P}(\vec{\theta})$ can be viewed as linear combinations of pose-blend shapes P learned from the datasets mentioned in \autoref{sec:benchmarks}. Each joint orientation $\theta$ is varied by rotating it 9 times relatively, therefore the iteration space is extended to the cardinality of $R_{n}(\vec{\theta})$ which is $\vert R_{n}(\vec{\theta}) \vert = K \times 9 = 207$. In \autoref{eq:Tf}, a second rotation set of the initial $T$ model is also initialised to define the vertex deviations and make them linear to P. Thus, the impact of an pose blend shape is neglected when a joint orientation is equal to the $T$ orientation.

\begin{equation}
\label{eq:Tf}
	\begin{split}
		T_{F}(\vec{\beta},\vec{\theta})&: \mathbb{R}^{\vert \vec{\theta} 			\vert \times \vert \vec{\beta} \vert} \mapsto \mathbb{R}^{3N} \\
		T_{F}(\vec{\beta},\vec{\theta}) &= T + B_{S}(\vec{\beta}) + B_{P}				(\vec{\theta}) \\
		&= T + B_{S}(\vec{\beta}) + \sum_{n=1}^{9K} (R_{n}(\vec{\theta}) - 				R_{n}(\vec{\theta}^{*}))P_{n};
	\end{split}
\end{equation}

Finally, the \emph{linear blend skinning} in \autoref{eq:lbs} can be rewritten by the formulations in \autoref{eq:smpl} and their application in \autoref{eq:Tf}. This \autoref{eq:lbs2} is formulated below, considering both $B_{P}$ and $B_{S}$ as matrices in which a vertex is indexed by i.

\begin{equation}
\label{eq:lbs2}
\bar{t}_{i}' = \sum_{k=1}^{K} w_{k,i} \cdot G_{k}'(\vec{\theta},J(\vec{\beta})) \cdot (\bar{t}_{i} + b_{S,i}(\vec{\beta}) +b_{P,i}(\vec{\theta}))
\end{equation}

