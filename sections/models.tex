\section{models in human estimation}
This section is dedicated to human models, which are besides being a representative concept, also central aspects on which iterative pose estimation approaches are based. While simple stick models, where the human structure is modeled by a few vertices, show the joints and limbs in 2d estimation, 3d models, originally built from simple primitives such as cylinders, have evolved into a vertex structure showing the shape, muscles and body structure of a human. To conclude, this section explained in detail the human body models commonly used in 3d estimation methods.

\subsection{Skinned Multi-Person Linear model SMPL}
\label{sec:SMPL}
In \emph{SMPL} \cite{smpl} a model $M(\vec{\beta},\vec{\theta},\phi)$ is learned from the \emph{Caesar} 3d scans explained in \autoref{sec:benchmarks}, that returns a mesh from the input. This formulation is also included in \autoref{eq:model}, where $\mathbb{R}^{3N}$ is a vector of \emph{N = 6890} vertices sculpturing the mesh. In this formula, $\vec{\beta}$ is a vector that contains the coefficients for the shape blend shapes $\mathbb{S}$. 
A single shape blend $S \in \mathbb{S}$ is a vector of vertex shifts corresponding to a particular feature of a human, e.g., height or weight. $\vec{\theta}$ is a vector of poses, e.g. the concatenated joint rotation in axis-angles representation, of all joints and the global rotation, which are also used as coefficients for pose blend shapes, and $\phi$ describes other learned parameters. 

\begin{equation}
\label{eq:model}
M(\vec{\beta},\vec{\theta},\phi) : \mathbb{R}^{\vert \vec{\theta} \vert \times \vert \vec{\beta} \vert} \mapsto \mathbb{R}^{3N}
\end{equation}

\emph{SMPL} is based on vertex skinning and blend shapes. A vertex changes its position depending on the motion of the associated joint. This displacement is controlled by assigned blend weights. A vector $T \in \mathbb{R}^{3N}$ of vertex positions describes a gender neutral initial human model, while a matrix $W \in \mathbb{R}^{N \times K}$ represents the blend weights per vertices and \emph{K = 23} joints. The joints that describe the human structure and form the skeleton are represented by 3d positions. Moreover, $T$ can be rearranged by the pose-blending function $B_{P}(\vec{\theta})$ according to the given pose parameter, leaving its initial shape unaffected, while $B_{S}(\vec{\beta})$ reshapes the identity model by its given shape blend coefficients. The two terms $B_{P}(\vec{\theta})$ and $B_{S}(\vec{\beta})$ thus serve as offsets to the initial mesh $T$, which are included in the sum $T_{F}(\vec{\beta},\vec{\theta})$. Finally, a skinning function $W(T_{F}(\vec{\beta},\vec{\theta}),J(\vec{\beta}),\vec{\theta},\mathcal{W})$ can be created containing the initial model $T$ transformed by $T_{F}$, the joint positions calculated from $J(\vec{\beta})$, the rotations $\vec{\theta}$ and the blending weights $\mathcal{W}$, shown in \autoref{eq:smpl}. Since $\beta$ affects the shape of a person and thus the position of his joints, the joint positions must be calculated after the shape displacements $B_{S}$ have been applied to the initial model $T$. The vertices in \emph{smpl} have a strict structure so that the position of a joint can be determined by the vertices in its region. Therefore, a joint regression matrix is learned to assign weights to these particular vertices that affect joint position, which were then linearly combined to define its location.

\begin{equation}
\label{eq:smpl}
M(\vec{\beta},\vec{\theta}) = W(T_{F}(\vec{\beta},\vec{\theta}),J(\vec{\beta}),\vec{\theta},\mathcal{W}) \mapsto \mathbb{R}^{3N}
\end{equation}

As mentioned previously, a model is determined of its initial state $T$ and its offsets in $T_{F}$. In $T_{F}(\vec{\beta},\vec{\theta})$, $T$ is considered as a known parameter, while the pose correctives $B_{P}(\vec{\theta})$ and the reshapes $B_{S}(\vec{\beta})$ are added. The shape of a single human can be viewed as a sum of shape blend shapes $S \in \mathbb{S}$ multiplied by their corresponding coefficients in $\beta$, describing the impact of the associated features on the resulting mesh. This concept is expressed in \autoref{eq:bs}.

\begin{equation}
\label{eq:bs}
B_{s}(\vec{\beta},\mathbb{S}) = \sum_{n=1}^{|\vec{\beta}|}\beta_{n} \cdot S_{n}
\end{equation}

With respect to \autoref{eq:Tf}, $B_{P}(\vec{\theta})$ can be viewed as linear combinations of pose blend shapes $\mathbb{P}$ learned from \emph{Caesar} dataset.

\begin{equation}
\label{eq:Tf}
	\begin{split}
		T_{F}(\vec{\beta},\vec{\theta})&: \mathbb{R}^{\vert \vec{\theta} 			\vert \times \vert \vec{\beta} \vert} \mapsto \mathbb{R}^{3N} \\
		T_{F}(\vec{\beta},\vec{\theta}) &= T + B_{S}(\vec{\beta}) + B_{P}				(\vec{\theta}) \\
		&= T + B_{S}(\vec{\beta}) + \sum_{n=1}^{9K} (R_{n}(\vec{\theta}) - 				R_{n}(\vec{\theta}^{*}))P_{n};
	\end{split}
\end{equation}

$\theta$ is a concatenation of all $K$ rotations of the joints plus the root orientation. These rotations are represented as axis-angles, which can be converted into $\mathbb{R}^{3\times3}$ rotation matrices using the \emph{Rodrigues} formula\footnote{\url{https://mathworld.wolfram.com/RodriguesRotationFormula.html}}. Each element of this matrix can be stacked into a 9d vector, applying it to $K$ joint yields a 207d vector. This entire concept is described by the function 
$R: \mathbb{R}^{|\vec{\theta}|} \mapsto \mathbb{R}^{9K}$. In \autoref{eq:Tf}, $R_{n}(\vec{\theta}^{*}))$ denotes the nth element of the 207d vector resulting from the residual position of $T$, while $R_{n}(\vec{\theta}))$ denotes the nth element of the entered pose rotations $\theta$. The effect of the learned pose-blend shapes thus depends on the deviation between $\theta$ and $\theta^{*}$ and is neglected if the nth joint orientation element corresponds to the initial orientation of the nth joint rotation element in $T$. \\

\autoref{eq:Tf} is used in skinning, in general to calculate a new vertex position based on the shape and pose parameters. A previous skinning approach is called \emph{linear blend skinning}, where each resulting vertex $\bar{t}_{i}'$ depends on the sum of joint weights $w_{k,i}$ multiplied by its previous position $\bar{t}_{i}$ and $G_{k}'$, as documented in \autoref{eq:lbs}. $G_{k}$ is the homogeneous world transformation of a joint k, where a rotation matrix is obtained by applying the Rodrigues formula and the translational part of the homogeneous matrix is given by the joint position. $G_{k}'$ is obtained by removing the transformation of $\theta^{*}$.  

\begin{equation}
\label{eq:lbs}
\bar{t}_{i}' = \sum_{k=1}^{K} w_{k,i} \cdot G_{k}'(\vec{\theta},J) \cdot \bar{t}_{i}
\end{equation}

Unfortunately, this approach is error-prone and requires additional refinement by the artists to reduce artifacts. To overcome these problems, the skinning method in \emph{SMPL} can be rewritten by the formulations in \autoref{eq:smpl} and their application in \autoref{eq:Tf}. This \autoref{eq:lbs2} is formulated below, considering both $b_{P}$ and $b_{S}$ as vectors, which offset the vertex position $\bar{t}_{i}$.

\begin{equation}
\label{eq:lbs2}
\bar{t}_{i}' = \sum_{k=1}^{K} w_{k,i} \cdot G_{k}'(\vec{\theta},J(\vec{\beta})) \cdot (\bar{t}_{i} + b_{S,i}(\vec{\beta}) + b_{P,i}(\vec{\theta}))
\end{equation}
