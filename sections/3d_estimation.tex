\section{3d pose estimation}
From a historical perspective, a 3d motion capture algorithm consists of 4 sequential processes: initialisation, tracking, pose estimation and recognition. Initialization involves both camera and model initialization, i.e. setting the camera calibration and finding a model that represents the subject and assigning its initial pose manually or automatically. Model-based approaches can be viewed iteratively, with each frame of the data source representing an iteration in which the initial pose is refined. Tracking is concerned with the relationship between the parts of the subject's body. This leads to segmentation of the subject from the background, representation changes, and establishing tracking in further images. \cite{summary80s}